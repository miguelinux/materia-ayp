% ex: ts=2 sw=2 sts=2 et filetype=tex
% SPDX-License-Identifier: CC-BY-SA-4.0

\section{Funciones}

\begin{frame}[c]{Funciones en Python}
  \vspace{\baselineskip}
  Una función es un bloque de código que solo se ejecuta cuando se llama.

  \vspace{\baselineskip}
  Puede pasar datos, conocidos como parámetros, a una función.

  \vspace{\baselineskip}
  Una función puede devolver datos como resultado. 
\end{frame}

\begin{frame}[fragile]
  \frametitle{Creando una función}

  En Python, una función se define usando la palabra clave
  \textcolor{codeKeyword}{def}:

  \vspace{\baselineskip}
  \begin{lstlisting}[language=Python]
  def mi_funcion():
    print("Hola desde una funcion")
  \end{lstlisting}
\end{frame}

\begin{frame}[fragile]
  \frametitle{Llamando a una función}

  Para llamar a una función, use el nombre de la función
  seguido de paréntesis: 

  \vspace{\baselineskip}
  \begin{lstlisting}[language=Python]
  def mi_funcion():     # esta es la definicion de la funcion
    print("Hola desde una funcion") # este es su cuerpo

  mi_funcion()  # aqui se manda a llamar
  \end{lstlisting}
\end{frame}

\begin{frame}[fragile]
  \frametitle{Argumentos}

  \vspace{\baselineskip}
  La información se puede pasar a funciones como \textbf{argumentos}.

  \vspace{\baselineskip}
  Los argumentos se especifican después del nombre de la función,
  \emph{entre paréntesis}. Puede agregar tantos argumentos como desee,
  solo sepárelos con una coma.

  \pausa
  \vspace{\baselineskip}
  El siguiente ejemplo tiene una función con un argumento (nombre).
  Cuando se llama a la función, pasamos un nombre, que se usa
  dentro de la función para imprimir el nombre completo:

  \begin{lstlisting}[language=Python]
  def mi_funcion(nombre):
    print(nombre + " Lopez")

  mi_funcion("Emilio")
  mi_funcion("Tomas")
  mi_funcion("Luis") 
  \end{lstlisting}
\end{frame}

\begin{frame}[c]{¿Parámetros o argumentos?}

  \begin{exampleblock}{}
    Los argumentos a menudo se acortan a \emph{args} en las
    documentaciones de Python.
  \end{exampleblock} 

  \pausa
  \vspace{\baselineskip}
  Los términos \textbf{parámetro} y \textbf{argumento} se pueden
  usar para lo mismo: información que se pasa a una función.

  \pausa
  \vspace{\baselineskip}
  \begin{block}{Desde la perspectiva de una función:}
    Un \textbf{parámetro} es la variable que aparece entre paréntesis en
    la \underline{definición de la función}.

    \vspace{\baselineskip}
    Un \textbf{argumento} es el valor que se envía a la función 
    \underline{cuando se llama}.
  \end{block} 
\end{frame}

\begin{frame}[fragile]
  \frametitle{Número de argumentos}

  De forma predeterminada, se debe llamar a una función con el
  número correcto de argumentos. Lo que significa que si una
  función espera 2 argumentos, debe llamar a la función con
  2 argumentos, ni más ni menos. 

  \vspace{\baselineskip}
  \begin{lstlisting}[language=Python]
  def mi_funcion(nombre, apellidos):
      print(nombre + " " + apellidos)

  mi_funcion("Julio", "Lopez") 
  \end{lstlisting}
\end{frame}

\begin{frame}[fragile]
  \frametitle{Número de argumentos}

  Si intenta llamar a la función con 1 o 3 argumentos, obtendrá un error:

  \vspace{\baselineskip}
  \begin{lstlisting}[language=Python]
  def mi_funcion(nombre, apellidos):
      print(nombre + " " + apellidos)

  mi_funcion("Julio") 
  \end{lstlisting}
\end{frame}

\begin{frame}[fragile]
  \frametitle{Argumentos arbitrarios, * argumentos}

  \vspace{\baselineskip}
  Si no sabe cuántos argumentos se pasarán a su función,
  agregue un \textbf{*} antes del nombre del parámetro en la
  definición de la función.

  \vspace{\baselineskip}
  De esta forma, la función recibirá una tupla de argumentos
  y podrá acceder a los elementos en consecuencia: 

  \vspace{\baselineskip}
  \begin{lstlisting}[language=Python]
  def mi_funcion(*ninios):
      print("El mas ninio mas joven es " + ninios[2])

  mi_funcion("Julio", "Luis", "Pedro")
  \end{lstlisting}

  \pausa
  \begin{exampleblock}{}
    Los argumentos arbitrarios a menudo se acortan a *args en
    la documentación de Python.
  \end{exampleblock}
\end{frame}

\begin{frame}[fragile]
  \frametitle{Argumentos con nombre de variables}

  También se puede enviar argumentos con la sintaxis \textbf{clave = valor}.

  \vspace{\baselineskip}
  De esta forma no importa el orden de los argumentos. 

  \vspace{\baselineskip}
  \begin{lstlisting}[language=Python]
  def mi_funcion(ninio3, ninio2, ninio1):
      print("El mas ninio mas joven es " + ninios3)

  mi_funcion(ninio1="Julio", ninio2="Luis", ninio3="Pedro")
  \end{lstlisting}
\end{frame}

\begin{frame}[fragile]
  \frametitle{Valores devueltos}

  Para permitir que una función devuelva un valor,
  use la declaración de \textcolor{codeKeyword}{return}:

  \vspace{\baselineskip}
  \begin{lstlisting}[language=Python]
  def mi_funcion(x):
      return 5 * x

  print(mi_funcion(3))
  print(mi_funcion(5))
  print(mi_funcion(9))
  \end{lstlisting}
\end{frame}

\begin{frame}[fragile]
  \frametitle{La declaración \textbf{pass}}

  Las definiciones de función no pueden estar vacías, pero si
  por alguna razón tiene una definición de función sin contenido,
  coloque la instrucción \textcolor{codeKeyword}{pass} para
  evitar errores. 

  \vspace{\baselineskip}
  \begin{lstlisting}[language=Python]
  def mi_funcion(x):
      pass
  \end{lstlisting}
\end{frame}

\begin{frame}[c]{Ejercicio}
  \vspace{\baselineskip}
  Realizar un programa en Python que represente un menú con las opciones
  siguientes:

  \vspace{\baselineskip}
  \begin{enumerate}
    \item Calcular el área de un rectángulo
    \item Calcular el área de un triangulo
    \item Calcular el área de un cuadrado
    \item Calcular el área de un circulo
    \item Salir
  \end{enumerate}


  \vspace{\baselineskip}
  Utilizar funciones para imprimir el menú, y para calcular cada una de las
  áreas. Las funciones para calcular áreas deben regresar su valor (usando
  return)
\end{frame}
