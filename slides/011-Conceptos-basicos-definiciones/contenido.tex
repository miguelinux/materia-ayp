% ex: ts=2 sw=2 sts=2 et filetype=tex
% SPDX-License-Identifier: CC-BY-SA-4.0
\begin{frame}
    \frametitle{Contenido}
    \tableofcontents
\end{frame}

\section{Definiciones básicas}

\begin{frame}[c]{Conceptos}
  \begin{itemize}
    \item Un algoritmo se pude escribir en diferentes idiomas o
      \textbf{lenguajes}.
    \pausa
    \item Un \textbf{programa} es un algoritmo escrito en \textbf{lenguaje
      máquina}.
    \pausa
    \item El \textbf{programador} se encarga de desarrollar los programas
      computacionales.
    \pausa
    \item Los programadores utilizan:
      \begin{itemize}
        \item Lenguajes de programación
        \pausa
        \item Compiladores
        \pausa
        \item Interpretes
        \pausa
        \item Enlazadores
        \pausa
        \item Depuradores (debuggers)
        \pausa
        \item Entornos de desarrollo integrados (IDEs)
      \end{itemize}
      para realizar sus programas.
  \end{itemize}
\end{frame}

\begin{frame}[c]{Conceptos}
  \begin{itemize}
    \item Para poder desarrollar un programa es necesario utilizar un
      \textbf{lenguaje de programación}.
    \pausa
    \item Cada lenguaje de programación define reglas de escritura:
      \textbf{Léxicas}, \textbf{sintácticas} y \textbf{semánticas}.
    \pausa
    \item Existen muchos lenguajes de programación:
      \begin{multicols}{3}
        \begin{itemize}
          \item C
          \pausa
          \item C++
          \pausa
          \item C\#
          \pausa
          \item Java
          \pausa
          \item JavaScript
          \pausa
          \item Python
          \pausa
          \item PHP
          \pausa
          \item Go
          \pausa
          \item Rust
        \end{itemize}
      \end{multicols}
  \end{itemize}
\end{frame}

\begin{frame}[c]{Conceptos}
  \begin{itemize}
    \item Las instrucciones que escribió el programador forman el
      \textbf{código fuente} (source code). Entonces, el código fuente es un
      algoritmo escrito utilizando un lenguaje de programación.

    \item Un \textbf{compilador} es un programa que se encarga de traducir
      el código fuente a un código objeto (obj), y que junto un enlazador
      (linker) genera un programa en lenguaje máquina directamente ejecutable.

    \item El \textbf{lenguaje máquina} es aquél que es entendido y ejecutado
      por una computadora. En Windows tienen la extinción \textbf{exe} y
      \textbf{dll}.

  \end{itemize}
\end{frame}
